\documentclass[12pt]{article}
\usepackage{fullpage,times,subfigure,fancyhdr}
\usepackage{amsfonts,amsmath,amssymb,amsthm}
\usepackage[pdftex]{color}
\usepackage{wrapfig,sidecap,subfigure}
\usepackage{textcomp}
\usepackage{verbatim}
\usepackage{times}
\usepackage{textcomp}
\usepackage{url}
\usepackage{mdwlist}
\usepackage{xspace}
\usepackage{algorithm,algorithmic}
\usepackage[pdftex]{graphicx}
\usepackage{mcode}
\usepackage[colorlinks=true,pagebackref,linkcolor=magenta]{hyperref}
\usepackage[sort&compress,comma,square,numbers]{natbib}
% \usepackage{hyperref}
% \usepackage{cite}

\pagestyle{fancy}
\oddsidemargin=0.0in 
\evensidemargin=0.0in
\textwidth=6.5in 
\headwidth=6.5in
\textheight=9in 
\headheight=12pt
\topmargin=-0.25in
\headsep=0.25in

\input{latex_commands}


\usepackage{wrapfig}
\usepackage{sidecap}

\title{\vspace{-75pt}Big Brain Benchmarks}
\author{}
\date{}

\begin{document}
{\vspace{-100pt}
\maketitle
}\setcounter{tocdepth}{2}
\tableofcontents

\begin{abstract}
\end{abstract}

\newpage
\section{Introduction}

\section{Preliminaries} \label{sec:methods}

\subsection{Pre-processing}


\subsubsection{CPAC}


\subsubsection{NIAK}


\subsubsection{Something else}


\subsection{Notation}



\subsection{Reproducibility and Extensibility}


Website for this work including leaderboards: \url{www.bigbrainbenchmarks.org}.

Python package for download data and running all algorithms: 

Direct link to download data:

\section{Inference Tasks}
\subsection{Testing}


\subsubsection{Data}

We use the data from \cite{Elkund2013} for testing, which is the 1000 Functional Connectome Project Data.

\subsubsection{Algorithms}

\begin{itemize} \itemsep0pt
	\item t-test:
	\item rank-sum test:
	\item wilcoxon:
	\item Lq-test: 
\end{itemize}


\subsubsection{Performance Metrics}


False positive and false negative rates as well as computational time.


\subsection{Classification}

\subsubsection{Data}

ABIDE \cite{abide}, autistic or not.

\subsubsection{Algorithms}

\begin{itemize} \itemsep0pt
	\item Linear Discriminant Analysis (LDA) $\circ$ embedding:
	\item Quadratic Discriminant Analysis (QDA) $\circ$ embedding:
	\item Support Vector Machine (SVM):
	\item Random Forest (RF):
	\item $k$-Nearest Neighbor: 
\end{itemize}

For embeddings, we consider:
\begin{itemize} \itemsep0pt
	\item PCA:
	\item Random Projections:
	\item Iterative Denoising Trees: 
\end{itemize}

\subsubsection{Performance Metrics}


\begin{itemize} \itemsep0pt
	\item Area Under Curve (AUC): Though see \cite{Hand2010}
	\item Npairs: 
\end{itemize}

\subsection{Regression}

\subsubsection{Data}

ABIDE \cite{abide} age.


\subsubsection{Algorithms}

\begin{itemize} \itemsep0pt
	\item Ridge Regression (RR):
	\item Linear with Total Variation Penalty (LTV): for spatial and temporal smoothing
	\item Support Vector Regression (SVR):
	\item $k$-Nearest Neighbor Regression: 
\end{itemize}


\subsubsection{Performance Metrics}


\subsection{Time-Series Prediction}

\subsubsection{Data}

\subsubsection{Algorithms}

\begin{itemize} \itemsep0pt
	\item Vector Auto-Regression (VAR):
	\item Kalman Filter-Smoother (KFS):
	\item Support Vector Regression (SVR): 
\end{itemize}

\subsubsection{Performance Metrics}

\begin{itemize} \itemsep0pt
	\item Integrated L2 error for $t$ time steps ahead: 
\end{itemize}



\subsection{Clustering}

Cameron wanted some unsupervised something in here.

\subsubsection{Data}

NKI Test-Retest Data.  We clustering time-series of each individual to parcellate the brain.

\subsubsection{Algorithms}

\begin{itemize} \itemsep0pt
	\item k-means:
	\item Spectral Clustering:
	\item METIS: 
\end{itemize}

\subsubsection{Performance Metrics}

Something like reliability, e.g.:

\begin{enumerate} \itemsep0pt
	\item run each clustering algorithm separately on each scan
	\item compute cluster similarity metric to compare each scan ($2 \times n$), where $n$ is the number of subjects
	\item rank each scan according to its ordering relative to the other scan for that individual
	\item cluster performance is the sum of ranks for each algorithm
\end{enumerate} 





\section{Results} \label{sec:results}

\subsection{Testing}

\subsection{Classification}

\subsection{Regression}

\subsection{Time-Series Prediction}


\subsection{Clustering}

\section{Discussion} \label{sec:disc}


\newpage
\small{
\bibliography{library}
\bibliographystyle{IEEEtran}
}


\end{document}
